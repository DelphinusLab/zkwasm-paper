%%
%% This is file `sample-acmsmall.tex',
%% generated with the docstrip utility.
%%
%% The original source files were:
%%
%% samples.dtx  (with options: `acmsmall')
%% 
%% IMPORTANT NOTICE:
%% 
%% For the copyright see the source file.
%% 
%% Any modified versions of this file must be renamed
%% with new filenames distinct from sample-acmsmall.tex.
%% 
%% For distribution of the original source see the terms
%% for copying and modification in the file samples.dtx.
%% 
%% This generated file may be distributed as long as the
%% original source files, as listed above, are part of the
%% same distribution. (The sources need not necessarily be
%% in the same archive or directory.)
%%
%% Commands for TeXCount
%TC:macro \cite [option:text,text]
%TC:macro \citep [option:text,text]
%TC:macro \citet [option:text,text]
%TC:envir table 0 1
%TC:envir table* 0 1
%TC:envir tabular [ignore] word
%TC:envir displaymath 0 word
%TC:envir math 0 word
%TC:envir comment 0 0
%%
%%
%% The first command in your LaTeX source must be the \documentclass command.
\documentclass[acmsmall]{acmart}
%% NOTE that a single column version is required for 
%% submission and peer review. This can be done by changing
%% the \doucmentclass[...]{acmart} in this template to 
%% \documentclass[manuscript,screen]{acmart}
%% 
%% To ensure 100% compatibility, please check the white list of
%% approved LaTeX packages to be used with the Master Article Template at
%% https://www.acm.org/publications/taps/whitelist-of-latex-packages 
%% before creating your document. The white list page provides 
%% information on how to submit additional LaTeX packages for 
%% review and adoption.
%% Fonts used in the template cannot be substituted; margin 
%% adjustments are not allowed.
%%
%% \BibTeX command to typeset BibTeX logo in the docs
\AtBeginDocument{%
  \providecommand\BibTeX{{%
    \normalfont B\kern-0.5em{\scshape i\kern-0.25em b}\kern-0.8em\TeX}}}

%% Rights management information.  This information is sent to you
%% when you complete the rights form.  These commands have SAMPLE
%% values in them; it is your responsibility as an author to replace
%% the commands and values with those provided to you when you
%% complete the rights form.
\setcopyright{acmcopyright}
\copyrightyear{2018}
\acmYear{2018}
\acmDOI{XXXXXXX.XXXXXXX}


%%
%% These commands are for a JOURNAL article.
\acmJournal{PACMPL}
\acmVolume{37}
\acmNumber{4}
\acmArticle{111}
\acmMonth{8}

%%
%% Submission ID.
%% Use this when submitting an article to a sponsored event. You'll
%% receive a unique submission ID from the organizers
%% of the event, and this ID should be used as the parameter to this command.
%%\acmSubmissionID{123-A56-BU3}

%%
%% For managing citations, it is recommended to use bibliography
%% files in BibTeX format.
%%
%% You can then either use BibTeX with the ACM-Reference-Format style,
%% or BibLaTeX with the acmnumeric or acmauthoryear sytles, that include
%% support for advanced citation of software artefact from the
%% biblatex-software package, also separately available on CTAN.
%%
%% Look at the sample-*-biblatex.tex files for templates showcasing
%% the biblatex styles.
%%

%%
%% The majority of ACM publications use numbered citations and
%% references.  The command \citestyle{authoryear} switches to the
%% "author year" style.
%%
%% If you are preparing content for an event
%% sponsored by ACM SIGGRAPH, you must use the "author year" style of
%% citations and references.
%% Uncommenting
%% the next command will enable that style.
%%\citestyle{acmauthoryear}
\usepackage{float}
%%
%% end of the preamble, start of the body of the document source.
\begin{document}

%%
%% The "title" command has an optional parameter,
%% allowing the author to define a "short title" to be used in page headers.
\title{ZKSNARK WASM Emulator}

%%
%% The "author" command and its associated commands are used to define
%% the authors and their affiliations.
%% Of note is the shared affiliation of the first two authors, and the
%% "authornote" and "authornotemark" commands
%% used to denote shared contribution to the research.
\author{Xin Gao}
\authornote{All authors contributed equally to this research.}
\email{xgao@zoyoe.com}
\orcid{1234-5678-9012}
\author{Hongfei Fu}
\email{jt002845@sjtu.edu.cn}
\author{Heng Zhang}
\author{Guoqiang Li}
\email{li.g@sjtu.edu.cn}
\orcid{0000-0001-9005-7112}
\author{Junyu Zhang}



%\authornotemark[1]
\affiliation{%
  \institution{Delphinus Lab.}
  \streetaddress{P.O. Box 1212}
  \city{Sydney}
  \state{NSW}
  \country{Australia}
  \postcode{43017-6221}
}

%\authornotemark[2]
\affiliation{%
  \institution{Shanghai Jiao Tong University}
  \streetaddress{800 Dongchuan Rd.}
  \city{Shanghai}
  \country{China}
  \postcode{200240}
}

%%
%% By default, the full list of authors will be used in the page
%% headers. Often, this list is too long, and will overlap
%% other information printed in the page headers. This command allows
%% the author to define a more concise list
%% of authors' names for this purpose.
\renewcommand{\shortauthors}{Sinka and Li, et al.}


%% commands
\newcommand{\zkwasm}{ZKWASM}
\newcommand{\fullstate}{$(\mathcal{F}, \mathcal{M}, \mathcal{S}_p, \mathcal{I}(\mathcal{C,\mathcal{H}}))$}
%%
%% The abstract is a short summary of the work to be presented in the
%% article.
\begin{abstract}
  zk-SNARK (zero-knowledge Succinct Non-interactive Argument of Knowledge) is a powerful proof system that allows efficient verification of the evaluation problem of polynomials. To further extending this technology to verify the evaluation of computer programs, we present ZKWASM, a ZKSNARK backed virtual machine that emulates the execution of web assembly bytecode and generates zero-knowledge-proofs for the emulation result. The proof generated by the \zkwasm virtual machine can then be used to convince an entity, with no leakage of confidential information, that the result of the emulation enforces the semantic specification of web assembly 
\end{abstract}

%%
%% The code below is generated by the tool at http://dl.acm.org/ccs.cfm.
%% Please copy and paste the code instead of the example below.
%%
\begin{CCSXML}
<ccs2012>
 <concept>
  <concept_id>10010520.10010553.10010562</concept_id>
  <concept_desc>Computer systems organization~Embedded systems</concept_desc>
  <concept_significance>500</concept_significance>
 </concept>
 <concept>
  <concept_id>10010520.10010575.10010755</concept_id>
  <concept_desc>Computer systems organization~Redundancy</concept_desc>
  <concept_significance>300</concept_significance>
 </concept>
 <concept>
  <concept_id>10010520.10010553.10010554</concept_id>
  <concept_desc>Computer systems organization~Robotics</concept_desc>
  <concept_significance>100</concept_significance>
 </concept>
 <concept>
  <concept_id>10003033.10003083.10003095</concept_id>
  <concept_desc>Networks~Network reliability</concept_desc>
  <concept_significance>100</concept_significance>
 </concept>
</ccs2012>
\end{CCSXML}

\ccsdesc[500]{Computer systems organization~Embedded systems}
\ccsdesc[300]{Computer systems organization~Redundancy}
\ccsdesc{Computer systems organization~Robotics}
\ccsdesc[100]{Networks~Network reliability}

%%
%% Keywords. The author(s) should pick words that accurately describe
%% the work being presented. Separate the keywords with commas.
\keywords{datasets, neural networks, gaze detection, text tagging}

\received{10 November 2022}
%\received[revised]{12 March 2009}
%\received[accepted]{5 June 2009}


%%
%% This command processes the author and affiliation and title
%% information and builds the first part of the formatted document.
\maketitle

\section{Introduction}
WASM (or WebAssembly) is an open standard binary code format close to assembly. Its initial objective is to provide an alternative to java-script and better performance in the current web ecosystems. Benefiting from its platform independence, front-end flexibility (can be compiled from the majority of languages including C, C++, assembly script, rust, etc.), good isolated runtime and speed that is close to native binary, its usage starts to arise in the distributed cloud and edge computing. Recently it has become a popular binary format for users to run customized functions on AWS Lambda, Open Yurt, AZURE, etc.

As with the technology of WASM runtime for cloud and edge computing shifts, security and privacy \cite{ pearson2009taking} issues emerge in scenarios that demand trustless \cite{wood2016trustless, chang2002trustless} computation and privacy computing \cite{xiao2012security,takabi2010security}. For instance, suppose that there is a voting hub hosted in the cloud to collect votes for proposals. The role of this service is to report the voting results to users while not leaking any information about any voters' choices. In this scenario, we would like a service that does not only provides the voting results but also provides proof to convince users that the provided results are calculated with predefined protocols (voting protocols). However since the service can not leak voters' personal choices, it should not reveal any voting ticket that is signed by a voter which makes it tricky to make the proof. 

Traditional ways to achieve trustlessness and privacy usually involve invasive changes to the source code of the service running on the cloud and those changes are usually applied in a case-by-case manner. In this work, instead of changing the code itself, we propose a novel approach by implementing \zkwasm, which is a WASM virtual machine that not only runs the WASM bytecode but also provides a zero-knowledge proof which is used to convince a verifier that the execution result is trustworthy.

The idea of \zkwasm\, is derived from ZKSNARK (Zero-Knowledge Succinct Non-Interactive Argument of Knowledge) which is a combination of SNARG (Succinct non-interactive arguments) and zero-knowledge proof. In general, the adoption of ZKSNARK usually requires implementing a program in arithmetic circuits (see Section \ref{chp:arith-circuits}) which forms a barrier for existing programs to leverage the power of it. Thus in this paper, we write the whole WASM virtual machine in ZKSNARK circuits so that existing WASM applications can benefit from ZKSNARK by simply running on the \zkwasm\, without any modification. Therefore, the cloud service provider can prove to any user that the computation result is computed honestly and no private information is leaked.

\smallskip\noindent\textbf{The Problem.}
To implement a ZKSNARK backed WASM virtual machine, we need to connect the implementation of WASM runtime with the proof system of ZKSNARK. In general, a ZKSNARK system is represented in arithmetic circuits (see Section \ref{chp:arith-circuits}) with polynomial constraints. Therefore we need to abstract the full imperative logic of a WASM virtual machine systematically and rewrite it into arithmetic circuits with constraints. Given two outputs, one is generated by emulating the WASM bytecode in WASM runtime that enforces the semantic of WASM specification, and the other satisfies the constraints imposed on the arithmetic circuits. If the circuits we write preserve the semantics, then these two outputs must be the same. Thus the proof of the ZKSNARKS derived from the circuits also shows that the output is valid as a result of emulating the bytecode in WASM runtime. 

\smallskip\noindent\textbf{Our Contribution.}
In this paper we systematically abstract the WASM runtime implementation and rewrite it into arithmetic circuits with constraints. By doing so, we have proposed and implemented the first zk-WASM virtual machine that supports WebAssembly specification. Moreover, by providing \zkwasm\, existing program compiled to WASM (without any modification) can then satisfy the privacy and trustless requirements that have recently emerged in cloud and edge computing.

\smallskip\noindent\textbf{Organization of the Paper.}
After a brief introduction to the basic ideas about how to connect a statefull virtual machine with SNARK in Section \ref{chp:preliminary}, we describe the basic building block and ingredients used to construct \zkwasm\, circuits in Section \ref{chp:constraint-system} and then present the circuits architecture in Section \ref{chp:architecture-circuits}. After the architecture is fixed, we list the circuit of each opcode of WASM in Section \ref{chp:instruction-circuits}. In Section \ref{chp:foreign} we discuss foreign instruction expansion which provides a way to extend the virtual machine for better performance and integration. In the end, we discuss the performance benchmark in Section \ref{chp:performance}.

\section{Preliminaries}
\label{chp:preliminary}
\subsection{Succinct Verification}
Succinct non-interactive arguments (SNARGs) enable verifying NP statements with lower complexity than required for classical NP verification. Traditionally, the focus has been on minimizing the length of such arguments; nowadays research has also focused on minimizing verification time, by drawing motivation from the problem of delegating computation.

Recent constructions of pre-processing SNARGs have achieved attractive features: they are publicly verifiable, proofs consist of only O(1) encrypted (or encoded) field elements, and verification is via arithmetic circuits of size linear in the NP statement. Additionally, these constructions seem to have “escaped the hegemony” of probabilistically-checkable proofs (PCPs) as a basic building block of succinct arguments.


\subsection{Polynomial Commitment Schemes}
PCS (Polynomial Commitment Schemes \cite{boneh2020halo-pcs,boneh2020efficient-pcs,kate2010polynomial-pcs}) is a powerful tool for constructing SNARK schemes for the statement of polynomial evaluation. PCS provides a way for the prover and verifier to commit to a polynomial $p$ and then open the commitment at any certain point (prove that the evaluation of $P$ a point $x$ is equal to a claimed value $v$). In this paper, without specification, we use KZG (Kate, Zaverucha and Goldberg) as our polynomial commitment scheme and the commitment formula for a polynomial $p$ is defined by $g^{p(\beta)}$ where $\beta$ is a random value negotiated by prover and verifier at the setup stage and $g$ is a point on a special elliptic curve.

\subsection{Stateless Programs as Arithmetic Circuits}
\label{chp:arith-circuits}
Once there exists a SNARK scheme for a prover to prove the evaluation of polynomials $p_i$ at $x_i$ to a verifier, we can then construct SNARK for the prover to prove that a program $P(X)$ has certain return value $v$ when input $X$ is provided. Many such constructions can be found in literature such as Groth, Sonic, Marlin, Plonk, etc. In this paper, we construct our ZK virtual machine using the proof system of Halo2, which is an extension of PLONK (See Section \ref{chp:constraint-system}). In PLONK setup, stateless programs can be encoded into a special format called arithmetic circuits.

An arithmetic circuit is a set of gates and each gate can have a set of inputs that need to be processed and a set of outputs that can be used as other gates' inputs. The gates can be connected together so as to carry out an arithmetic algorithm. In the end, the outputs of the circuit is the result of the algorithm. For example, suppose that we need to represent a sum algorithm that can be constructed by a list of add gates as in Figure \ref{fig:sum-gates}.

\begin{figure}[!ht]
\centerline{
\includegraphics[scale=0.8]{figs/arithment-circuit.png}
}
\caption{Arithment Circuit of Sum}\label{fig:sum-gates}
\end{figure}

Now we have connected stateless programs to arithmetic circuits, and we know how to construct SNARKS of polynomial evaluation using PCS. Therefore it remains to establish a connection between arithmetic circuits with polynomials. This can be achieved by using interpolation techniques and copy constraints in PLONKish (see Section \ref{chp:constraint-system}).

%PCS 是polynomialbase的,constraint base ---> PCS
\subsection{Connecting Stateful Virtual Machine with Arithmetic Circuits}
\label{chp:encode-state-in-circuits}
Now we need to make a setup further. Instead of constructing a SNARK scheme for stateless programs, we would like to construct a SNARK scheme for a stateful virtual machine. We do not construct such SNARK from scratch; we construct it from the knowing ingredients which are the arithmetic circuits in Section \ref{chp:arith-circuits}. 

To start with, we establish the connection between a virtual machine and a program by treating the virtual machine as a program that generates a list of state transitions and each transition is defined as a monad function between states. We denote the transition function as $T_i$ which takes an input $s:State$ and outputs a new state $s':S$. The type of state $\mathcal{S}$ is defined as a tuple of \fullstate \, where $\mathcal{F}$ is the calling frame, $\mathcal{M}$ is the memory state, $\mathcal{S}_p$ is the stack and $\mathcal{I}$ is the image which contains code image $\mathcal{C}$ and initial memory $\mathcal{H}$.

Also for an instruction $op$ at address $addr$ in the image $\mathcal{I}$ of a WASM binary file, we define the transition semantic of $op$ to be a pair of $(t^{addr}_{op}, c^{addr}_{op})$ where $t^{address}_{op}$ is a state transition function and $c^{addr}_{op}$ is the control flow function from $S$ to the address of next instruction.

To define a valid sequence of transitions $T_i$, we first require that $T_0$ enforces the transition semantic of the instruction at the entry point $iaddr$ and for all $addr, s, k$, $T_k(s) = t^{iaddr}_{op}(s) \rightarrow T_{k+1} = t_{op'}^{iaddr'}$ where $iaddr' = c_{op}^{iaddr}(s)$ and $op'$ is the opcode of the instruction at $'iaddr$.

Second, with this setup we can map the execution of an executable image $I$ in a virtual machine $\mathcal{V}_m$ to a sequence of transition function $T_i$ over an initial state $s_0$. By denoting $s_i = T_i (T_{i-1}(\cdots T_0(s_0)))$ and $s_e$ to be the last state of the transition sequence, we require the final opcode of $T_{e}$ is returned and the depth of the calling frame of $s_e$ is zero $\mathcal{F}(s_e).depth = 0$.

We say a system of arithmetic circuits $C$ of transitions is equivalent to a WASM virtual machine if for any given entry point $iaddr_0$ there exists a unique sequence of $T_i$ satisfies $C$ and $T_0 = c^{iaddr_0}_{op}$.

%Stateless -> Statefull 的虚拟机:monadic function while s is hash
\subsection{Leverage zero-knowledge in ZKWASM}
A \zksnark\, is a SNARK scheme that provides a way for a prover to prove statements without leaking any information. When we construct the above SNARK scheme for a virtual machine in a zero-knowledge way, then we create a ZK Virtual machine that can prove the execution of certain program image without leaking any information.  This feature makes the ZK virtual machine extremely useful in scenarios where the prover would like to prove that certain output is calculated from the execution of a particular program image but does not want to leak the data used.
 

\section{Basic \zkwasm\, building blocks}
As described in Section \ref{chp:encode-state-in-circuits}, arithmetic circuit is crucial in connecting program execution with SNARKS of polynomial evaluation. We use Halo2 proof system to constructing circuits for \zkwasm, therefore in this section we will give a brief of the Halo2 proof system and then introduce some basic techniques we use in \zkwasm.
\subsection{Brief of Halo2 Proof System}
\label{chp:constraint-system}
Halo2's arithmetic system is an extension of PLONK that supports custom arithmetic gates and polynomial lookup. Below we represents Halo's circuits by matrix of values with constraints on a row basis. Regarding a particular row (without loss of generality, we use $cur$ as the index of this particular row), we use the notation $r_i.(cur)$ to denote the cell of column $i$ in that row and use the notation $r_i.(cur + n)$ to denote the cell of $(k+n)$th row of column $i$. With this notation, we can define constraint system of a circuit matrix by equations of cells of each row and their siblings. For example, the circuit matrix of sum in Figure \ref{fig:sum-gates} can be constructed as follows.

\begin{table}[!h]
\begin{center}
\begin{tabular}{ | c | c | c |}
  \hline
  s & acc & operand \\ 
  \hline
 1 & $sum_0 = 0$ & $v_0$\\
 \hline
 1 & $sum_1$ & $v_1$\\
 \hline
 1 & $\cdots$ & $\cdots$\\
 \hline
 0 & $sum_k$ & $nill$\\
 \hline
\end{tabular}
\caption{circuit matrix of sum}
\label{tbl:sum-table}
\end{center}
\end{table}

\noindent while the constraint system enforced on each row is
\[
 C(cur) = \begin{cases}
     &s.(cur) \times (acc.(cur) + operand.(cur) - acc.(cur+1)) = 0 \\
     &s.(cur) \times (1-s.(cur)) = 0
 \end{cases}
\]
\begin{remark}
Notice that the first constraint makes sure that add gate (see Figure \ref{fig:sum-gates}) is applied to each row except the last row and the second constraint enforced that $s$ is either $1$ or $0$).
\end{remark}

\begin{definition}[Arithmetic Circuit]
Arithmetic Circuit is a $n \times m$ matrix with $m$ columns and $n$ rows equipped with a constraint system $C$ that for each row $cur$ in the matrix $C(cur)$ holds.
\end{definition}

There are two ways to define constraint in constraint system $C$. One way is using polynomial equations of cells and the other is using polynomial lookup. Polynomial lookup is a special constraint that can enforce an expression $expr$ of cells belongs to an existing table $T$. More precisely we say $expr \in T$ if and only if $plookup(T, expr) = 0$. 

%From a circuit description we can generate a proving key and a verification key, which are needed for the operations of proving and verification for that circuit.

\subsection{Representing Basic types in Halo2 Constraint System}
Recall that the polynomial commitment can prove a set of polynomials $p_i(x)$ has evaluation $v_i$ at $x_i$. To prove a arithmetic circuit matrix with constraint $C$ holds in Section \ref{chp:constraint-system}, the Plonkish proof system interpolates each column $c_i$ into polynomials $c_i(x)$ such that $c_i(j) = c_{ij}$ and then uses KCG commitment scheme to prove $\mathcal{C}(c_i(x)) = 0$ holds for all $x=1,2,3,\cdots$. 

However, to use KZG commitment scheme on polynomial $c_i$, we require $c_i(x) \in F$ where $F$ is the scalar field of some elliptic curve $\mathbf{C}$, therefore each $c_i(j)$ is in the scalar field $\mathbf{F}$ of elliptic curve $\mathbf{C}$ in Halo2's arithmetic circuit system. However, since the basic types in webassembly is i64 and i32 which does not match the number field $\mathbf{F}$ in Halo2, we need to add a constraint $x<2^{32}$ (or $x< 2^{64}$) to represent a variable $x$ with type $i32$ or $i64$. In \zkwasm, we use $\mathbf{T_N}$ to denote a table contains elements from $0$ to $2^N-1$ and then we uses polynomial lookup to prove that all values of a column $c_i$ are less than $2^N-1$ by $plookup(\mathbf{T_N}, c_i(j)) = 0$. Sometimes $N$ is large (e.g. 64) and $\mathbf{T_N}$ becomes too big. In such scenario we will decompose a i64 into several parts and prove that each parts are less then $2^8-1$. Below we use the notation $x \in \mathbf{T_N}$ to denote $x < 2^N$ and omit the details of decompose $x$ into small pieces when necessary.

\subsection{Representing Map using Polynomial Lookup of Tables}
\label{chp:map-repr}
Other than specifying range of columns, another usage of polynomial lookup is that we can encode state of key-value map into tables and using polynomial lookup to specify the semantics of getting a value of a certain key in a map. 

Here is an example. Recall that we represent state of \zkwasm\, by \fullstate \, where $\mathcal{C}$ and $\mathcal{H}$ are fixed by the WASM image. We encode state $\mathcal{C}$ and $\mathcal{H}$ in tables $\mathbf{T}_\mathcal{C} : Addr \times Opcode$ and $\mathbf{T}_\mathcal{H}: Addr \times U64$. Therefore we can use the polynomial lookup to specify the semantic of getting opcode $op$ at address $addr$ in $\mathcal{C}$ by $(addr, op) \in \mathbf{T}_\mathcal{C}$ and specify the semantic of WASM of getting the initial byte data $v$ at address $addr$ in $\mathcal{H}$ by $\exists d, (d, addr \div 8) \in \mathbf{T}_\mathcal{H} \wedge v = (d \gg (addr \% 8)) \% 256$.

\subsection{Representing Math Semantic as Arithmetic Circuits}
According to the WASM specification, the semantic of opcodes are usually defined as mathematical equations and state transformation. When implementing \zkwasm \, we need to construct arithmetic circuits in Halo2 such that the semantic of the opcodes are enforced. For example, suppose that the opcode $div_u$ (division of unsigned int) has the following semantic:
\[ div_u(a, b) = (a - a \bmod b) \div b \] 
Then it follows that to write the above mathematical definition into polynomial constraints we need to introduce intermediate witness $r$ such that the above semantics can be rewritten as follows:
\begin{equation}
\begin{cases}
    a = div_u(a,b) * b + r \\
    r < b
\end{cases}
\end{equation}
However, since $r$ and $b$ are in $\mathbf{F}$, it needs more work to represent $r < b$ in to polynomial constraints. Fortunately, in \zkwasm, we uses range check to constraint $r$ and $b$ within 64 bits, the above constraints can be further rewritten into the following polynomial constraints with one more extra witness $k$: 
\begin{equation}
\begin{cases}
    a = div_u(a,b) * b + r \\
    b = r + k \\
    a, r, b, k\in T_{64}, 
\end{cases}
\end{equation}
When dealing with opcode that has more complicated mathematical semantics, we need a way to formally prove that the derived constraints represents the same semantic. In \zkwasm, we uses Z3 to formally check that the mathematical definition is refined to the arithmetic circuits correctly.

\subsection{Representing Dynamic State using Polynomial Lookup Tables}
Given a program, suppose that the program is a sequence of state transition function $\{t_i\}$ and each transition might read or write finitely many key-value pairs in the state. Also suppose that each read or write of $\{t_i\}$ is ordered in a sequence $\{t_i^k\}$. We denote the access log $L_{tid}$ of $t_i^k$ to be a tuple of $(tid, accessType, address, value)$ such that each access log has the following semantic:
\begin{itemize}
    \item $(tid, init, addr, v ) := t_{tid}(s) := s.addr = v$
    \item $(tid, write, addr, v) := t_{tid}(s) := s.addr = v$
    \item $(tid, read, addr, v) := t_{tid}(s) := return \,\, s.addr$ and $t_{tid}(s) = v$
\end{itemize}
%Notice that the above definition has the following \emph{Reduction rule of read}.
%    \[
%    \begin{split}
%    init(addr, v) \circ read(addr') = \lambda s, \begin{cases}
%        v\,\, \textnormal{(if $addr == addr'$)} \\
%        read(addr, s)\,\, \textnormal{(if $addr' \neq addr$)}
%    \end{cases}\\
%    write(addr, v) \circ read(addr') = \lambda s, \begin{cases}
%        v\,\, \textnormal{(if $addr == addr'$)} \\
%        read(addr, s)\,\, \textnormal{(if $addr' \neq addr$)}
%    \end{cases}
%    \end{split}
%    \]
We can group the log by their access address and represent the above properties into a arithmetic circuit as in Table \ref{tbl:rw-table}.
\begin{table}[!h]
\begin{center}
\begin{tabular}{ | c | c | c | c | }
  \hline
  address & tid & accessType & value \\ 
  \hline
 $addr_1$ & $tid_1$ &  $acc_1$ & $v_1$ \\  
 $addr_1$ & $tid_2$ &  $acc_2$ & $v_2$ \\
  $addr_1$ & $tid_3$ &  $acc_3$ & $v_3$ \\  
 \hline
 $addr_2$ & $tid_4$ &  $acc_3$ & $v_4$ \\  
 $addr_2$ & $tid_5$ & $acc_4$ & $v_5$ \\
 \hline
 $\cdots$ & $tid_k$ & $acc_k$ & $v_k$ \\
 \hline
\end{tabular}
\caption{memory access table}
\label{tbl:rw-table}
\end{center}
\end{table}

\noindent Motivated by Table \ref{tbl:rw-table}, we can define a circuit with constraints of each row as follows (see Equation \ref{eq:rw-constraints}),
\begin{equation}
\label{eq:rw-constraints}
\begin{split}
    &r(cur).address \equiv r(next).address \rightarrow r(cur).tid \le r(next).tid \\
    &r(cur).address  \le r(next.address) \\
    &r(next).accessType \equiv read \rightarrow r(next).value \equiv r(cur).value \\
    &r(cur).address \neq r(prev).address \rightarrow r(cur).accessType \equiv init
\end{split}
\end{equation}
and then establish a one-to-one correspondence between a valid memory access log sequence and a table which satisfy Equation \ref{eq:rw-constraints}.
\begin{theorem}
\label{thm:one-one-rw-1}
Give an valid memory access log ${L_i}$, then there exists a unique table satisfy the above constraints.
\end{theorem}
\begin{proof}
\end{proof}
\begin{theorem}
\label{thm:one-one-rw-2}
Give an access log table $T$ that satisfy \ref{eq:rw-constraints}, then there exists a unique valid memory access log sequence $L_i$ such that $L_i.tid = T_k.tid$ and length of $L_i$ is equal to the length of $T$.
\end{theorem}
\begin{proof}
\end{proof}
When implementing \zkwasm, we encodes the memory and stack changes into access log tables of memory (or stack) access logs table $T$ defined by Constraints \ref{eq:rw-constraints}. By doing so we can enforce the result $v$ of state read instructions $t_i.read(addr)$ by checking $(t_i, read, addr, v) \in T$.


\section{overview of \zkwasm Virtual Machine}
\subsection{Workflow of \zkwasm}
As a ZKSNARK virtual machine, \zkwasm\, takes in three inputs that are a WASM image $I$, its entry point $E$ and an $IO$ (stdin, stdout) firmware. After emulating the execution of $I$ start with $E$ under $IO::stdin$, \zkwasm\, generates $IO::stdout$ with a ZKSNARK proof which proves that $IO::stdout$ is a valid output (the execution enforces the semantics of WASM specification).\\

\noindent\emph{Image Setup.}
Defined by the WASM specification, a WASM image $I$ is divided into sections. Among them, there are sections that do not affect the execution of WASM (custom section, type section, export section, data count section) and sections that decide the execution semantics (data section, code section, module section). At the image setup stage, we encode the code section into lookup table $T_\mathcal{I}$ and data section into the lookup table $T_\mathcal{H}$. These two tables will be used to enforce that each instruction in the execution trace is a valid instruction and that all the initialization of the memory access log table complies with the initial data section of image $I$.\\

\noindent\emph{Generate the Execution Trace.}
An execution trace is a list of execution log $L_i$ that each $L_i$ is an execution log of of the $i$th instruction during the execution of $(I, E, IO)$. We say an execution trace is valid if each $L_i$ is derived from the semantic of its previous instruction $I_{i-1}$, the current instruction $I_{i}$ and the current state \fullstate.\\

\noindent\emph{Synthesis Circuit.}
Once a valid execution trace is generated, it can be used to fill our main execution circuit $\mathcal{T}$, together with three lookup tables $T_\mathcal{F}$ (calling frame table), $T_\mathcal{M}$ (memory access log table) and $T_{\mathcal{S}_p}$ (stack access log table). \\

\noindent\emph{Generate ZK Proof.}
After all the circuits are synthesised, we can generate a ZKSNARK proof via Halo2's proof system. The proof can be used to prove that the execution trace and its output are valid.

\subsection{Runtime Architecture of \zkwasm}
\noindent\emph{Compiler}
\noindent\emph{Wasmi}
\noindent\emph{Synthesizer}
\noindent\emph{Prover}
%\noindent\emph{verification}

%The aim of any proof system is to be able to prove interesting mathematical or cryptographic statements.

%Typically, in a given protocol we will want to prove families of statements that differ in their public inputs. The prover will also need to show that they know some private inputs that make the statement hold.

%To do this we write down a relation, R, that specifies which combinations of public and private inputs are valid.

%The terminology above is intended to be aligned with the ZKProof Community Reference.

%To be precise, we should distinguish between the relation R, and its implementation to be used in a proof system. We call the latter a circuit.

%The language that we use to express circuits for a particular proof system is called an arithmetization. Usually, an arithmetization will define circuits in terms of polynomial constraints on variables over a field.

%The process of expressing a particular relation as a circuit is also sometimes called "arithmetization", but we'll avoid that usage.

%To create a proof of a statement, the prover will need to know the private inputs, and also intermediate values, called advice values, that are used by the circuit.

%We assume that we can compute advice values efficiently from the private and public inputs. The particular advice values will depend on how we write the circuit, not only on the high-level statement.

%The private inputs and advice values are collectively called a witness.

%Some authors use "witness" as just a synonym for private inputs. But in our usage, a witness includes advice, i.e. it includes all values that the prover supplies to the circuit.

%For example, suppose that we want to prove knowledge of a preimage x of a hash function H for a digest y:

%The private input would be the preimage x.

%The public input would be the digest y.

%The relation would be {(x,y):H(x)=y}.

%For a particular public input Y, the statement would be: {(x):H(x)=Y}.

%The advice would be all of the intermediate values in the circuit implementing the hash function. The witness would be x and the advice.

%A Non-interactive Argument allows a prover to create a proof for a given statement and witness. The proof is data that can be used to convince a verifier that there exists a witness for which the statement holds. The security property that such proofs cannot falsely convince a verifier is called soundness.

%A Non-interactive Argument of Knowledge (NARK) further convinces the verifier that the prover knew a witness for which the statement holds. This security property is called knowledge soundness, and it implies soundness.

%In practice knowledge soundness is more useful for cryptographic protocols than soundness: if we are interested in whether Alice holds a secret key in some protocol, say, we need Alice to prove that she knows the key, not just that it exists.

%Knowledge soundness is formalized by saying that an extractor, which can observe precisely how the proof is generated, must be able to compute the witness.

%This property is subtle given that proofs can be malleable. That is, depending on the proof system it may be possible to take an existing proof (or set of proofs) and, without knowing the witness(es), modify it/them to produce a distinct proof of the same or a related statement. Higher-level protocols that use malleable proof systems need to take this into account.

%Even without malleability, proofs can also potentially be replayed. For instance, we would not want Alice in our example to be able to present a proof generated by someone else, and have that be taken as a demonstration that she knew the key.

%If a proof yields no information about the witness (other than that a witness exists and was known to the prover), then we say that the proof system is zero knowledge.

%If a proof system produces short proofs —i.e. of length polylogarithmic in the circuit size— then we say that it is succinct. A succinct NARK is called a SNARK (Succinct Non-Interactive Argument of Knowledge).

%By this definition, a SNARK need not have verification time polylogarithmic in the circuit size. Some papers use the term efficient to describe a SNARK with that property, but we'll avoid that term since it's ambiguous for SNARKs that support amortized or recursive verification, which we'll get to later.

%A zk-SNARK is a zero-knowledge SNARK.

\section{\zkwasm\ Architecture Circuits}
\label{chp:architecture-circuits}

As we have prepared our circuit building blocks in Section \ref{chp:build-blocks}, we start constructing the main circuits involved in \zkwasm. We will first describe the workflow of \zkwasm\, by splitting it into four stages to give a big picture of how different circuits (see Figure \ref{fig:arch-circuits}) interplay with each other and then we will present the details of each circuit.

\begin{figure}[!ht]
\centerline{
\includegraphics[scale=0.7]{figs/arch-circuits.png}
}
\caption{Architecture circuits}\label{fig:arch-circuits}
\end{figure}


\noindent\emph{Step 1: Image Setup.}
Defined by the WASM specification, a WASM image $\mathbf{I}$ is divided into sections. Among them, there are sections that do not affect the execution of WASM (custom section, type section, export section, data count section) and sections that decide the execution semantics (initial memory section, code section, global data section). At the image setup stage, we encode the code section into the lookup table $\mathbf{T}_\mathbf{I}$ and the data section into the lookup table $\mathbf{T}_\mathbf{H}$. These two tables will be used to enforce that each instruction in the execution trace is a valid instruction and that all the initialization of the memory access log table complies with the initial data section of image $I$.\\

\noindent\emph{Step 2: Execution Trace Generation.}
Recall that a valid execution trace is a sequence of transition functions $\left[t_0, t_1, \cdots\right]$ such that each $t_i$ is related to the 
$i$th instruction during the execution of $(\mathbf{I}, \mathbf{E}, \mathbf{IO})$. We uses the standard WASM run-time interpreter to generate ${t_i}$ that is valid as defined in Definition \ref{def:valid-trace}.
\begin{remark}
We do not require the WASM run-time interpreter to be a trust component since if it generates an invalid sequence, the constraints of the Execution Circuit fail because our Execution Circuit enforces the semantics of each instruction.
\end{remark}

\smallskip\noindent\emph{Step 3: Synthesis Circuits.}
Once a valid execution trace is generated, it can be used to fill our main execution circuit $\mathbf{T}_\mathcal{E}$, together with other lookup tables $\mathbf{T}_\mathcal{F}$ (calling frame table), $\mathbf{T}_\mathcal{M}$ (memory access log table), $\mathbf{T}_\mathcal{G}$ (global access log table) and $\mathbf{T}_{\mathcal{SP}}$ (stack access log table). \\

\noindent\emph{Step 4: Proof Generation.}
After all the circuits are synthesised, we can generate a ZKSNARK proof via Halo2's proof system. The proof can be used to prove that the execution trace and its output are valid.

\subsection{Setup Circuits}
Setup circuits are filled by the \zkwasm\, compiler component and its purpose is to provide lookup tables $\mathbf{T}_\mathcal{C}$, $\mathbf{T}_\mathcal{H}$, $\mathbf{T}_\mathcal{G}$ that encode code section, initial memory section and global data section.\\

\noindent\emph{Code Section.}
The elementary items in the code section are $opcode$s of instructions that are grouped in a tree-like hierarchy. Each instruction can be indexed by $moid$ (modular id), $mmid$ (memory block instance id), $fid$ (function id) and $iid$ (offset of the instruction in a particular function). We denote $iaddr$ to be the tuple of $(moid, mmid, fid, iid)$ and represent the code section as a map from $iaddr$ to $opcode$. Using the technique in Section \ref{chp:map-repr}, it is equivalent to encoding the code section into $\mathbf{T}_\mathcal{C}$ (see Table \ref{tbl:code-table}). Code table $\mathbf{T}_\mathcal{C}$ is later used to constrain entries in execution table $\mathbf{T}_\mathcal{E}$ (see Section \ref{chp:ex-table}) such that if $e \in \mathbf{T}_\mathcal{E}$ then $(e.iaddr, e.opcode)$ must also in $\mathbf{T}_\mathcal{C}$. 

\begin{table}[!h]
\begin{center}
\caption{Code table}
\label{tbl:code-table}
\begin{tabular}{ | c | c | c | c | c | }
  \hline
  moid & mmid & fid & iid & opcode \\
  \hline
  $0x00$ & $0x01$ & $0x01$ & $0x00$ & $add$ \\
  \hline
  $0x00$ & $0x01$ & $0x01$ & $0x01$ & $sub$ \\ 
  \hline
  $\cdots$ & $\cdots$ & $\cdots$ & $\cdots$ & sub \\ 
  \hline
\end{tabular}

\end{center}
\end{table}

\noindent\emph{Initial Memory \& Global Data.}
The element items in the memory section of WASM image are unsigned 64 bit words (u64). The address of each $u64$ word can be indexed by $mmid$ and $offset$. Besides the value, memory can have types that are either mutable or immutable. Thus the memory section can be represented as a map from $(mmid, offset)$ to $(value, isMutable)$. Similarly, using the technique in Section \ref{chp:map-repr}, we can encode the initial memory section into $\mathbf{T}_\mathcal{H}$. Similar to the init memory section, the global data section contains variable instances that can be shared between different modules which can also be represented as a map from $(mmid, offset)$ to $(value, isMutable)$. Thus we merge two tables into one and use $ltype = Memory \,|\, Global$ to distinguish them (see Table \ref{tbl:init-memory-table} for an example of $\mathbf{T}_{\mathcal{H}}$).
\begin{table}[!h]
\begin{center}
\caption{Initial memory table}
\label{tbl:init-memory-table}
\begin{tabular}{ | c | c | c | c | c | }
  \hline
  $ltype$ & $mmid$ & \emph{offset} & $value$ & $isMutable$ \\
  \hline
  $Heap$ & $mmid_0$ & $1$ & $0x01$ & $true$ \\
  \hline
  $Heap$ & $mmid_1$ & $1$ & $0x01$ & $true$ \\
  \hline
  $Global$ & $mmid_2$ & $1$ & $0x01$ & $true$ \\ 
  \hline
  $Global$ & $mmid_3$ & $1$ & $0x01$ & $false$ \\ 
  \hline
\end{tabular}

\end{center}
\end{table}

We use $\mathbf{T}_\mathcal{H}$ to constrains entries in the memory access log table $\mathbf{T}_\mathcal{M}$ (see Table \ref{tbl:rw-table}) so that $\forall e, e\in \mathbf{T}_\mathcal{M} \wedge e.accessType = Init \rightarrow (e.iaddr, value) \in \mathbf{T}_\mathcal{H}$. The meaning of this constraint is that for each init access log in $\mathbf{T}_\mathcal{M}$ it must be defined in the initial memory section or global data section.


\subsection{Execution Trace Circuits}
\label{chp:ex-table}
Execution Trace Circuits are used to constraint the execution trace $\left[t_0,t_1,t_2,\cdots\right]$ (see Section \ref{chp:exec-trace}) emulated from WASMI (WASM interpreter). Each trace element is related to an instruction in the code table $\mathbf{T}_\mathcal{C}$ and has a predefined semantic based on the opcode. The semantics of a WASM opcode is defined based on its parameters derived from the stack and the micro operations. First, since WASM is a stack machine, we define the \emph{operands} of an opcode $op$ to be
\[
operands(op) = p_0, p_1, p_2 \cdots, p_k
\]
where $p_i$ are values on the stack and $p_i = stack[sp+i]$.
Second, we define the semantics of $op$ by a sequence of microoperations
\[
mop_i = \begin{cases}
    w_i = load(ltype, addr)\, \textnormal{ where $addr \in \{p_1, p_2, \cdots, p_k, w_0, w_1, \cdots, w_{i-1}\}$}\\
    write(ltype, addr, v)\, \textnormal{ where $addr, v \in \{p_1, p_2, \cdots, p_k, w_0, w_1, \cdots, w_{i-1}\}$}\\
    w_i = arith(p_1, p_2,\cdots, p_k, w_0, w_1, \cdots, w_{i-1}); \\
    \textit{FALLTHOUGH}; \\
    GOTO(iaddr); \\
    if \, b\, then\, \{mop_{i+1}, \cdots mop_{j}\} \,else\, \{mop_{j+1},\cdots\}.
    \end{cases}
\]
When filling execution trace into the execution circuit, we arrange the instruction into small blocks (see Table \ref{tbl:ex-table}) of the execution circuit such that each block represents an instruction. Within each block, we use the $start$ column to indicate whether this row is the start of a new instruction block and put $op$ and $mop$ in the opcode column. In the address column, we push all used addresses and the first row is the instruction address of this instruction in $\mathbf{T}_\mathcal{C}$ and in the $sp$ column we record all the changes of stack pointer.
\begin{table}[!h]
\begin{center}
\caption{Execution table}
\label{tbl:ex-table}
\begin{tabular}{ | c | c | c | c | c | c | c | c | c | c | }
  \hline
  start & opcode & bit cell & state & aux & $address \in T_{I}$ & $sp$ & u64 cell \\ 
  \hline
   true & $op$ & $b_0$ & $tid_0$ & $aux$ & $iaddr_0$ & $sp$ & $w_0$ \\ 
 \hline
   0 & $mop_0$ & $b_1$ & $frame$ & $aux_0$ & $addr_0$ & $\cdots$ & $w_1$ \\ 
 \hline
   0 & $mop_1$ & $b_2$ & $..$ & $aux_1$ & $addr_1$ & $\cdots$ & $w_2$ \\ 
 \hline 
  0 & $mop_2$ & $b_3$ & $s_3$ & $aux_2$ & $addr_2$ & $\cdots$ & $w_3$ \\ 
 \hline
   $\cdots$ & $\cdots$ & $\cdots$ & $\cdots$ & $\cdots$ & $\cdots$ & $\cdots$ & $\cdots$ \\ 
 \hline
   true & $op_1$ & $b$ & $tid_1$ & $aux$ & $iaddr_1$ & sp' & $w$ \\ 
 \hline
   $\cdots$ & $\cdots$ & $\cdots$ & $\cdots$ & $\cdots$ & $\cdots$ & $\cdots$ & $\cdots$ \\
 \hline
 \hline
\end{tabular}

\end{center}
\end{table}

\smallskip Although different opcodes might have different semantics thus different $mop_k$, $addr_i$, etc. There are some common constraints that we need to enforce in the execution circuit. First. we need to enforce that each instruction exists in the code section, thus $(iaddr, opcode) \in \mathbf{T}_\mathcal{C}$. Second, suppose that $operand$ $p_i$ is got from stack pointer $sp$ as a result of $mop_k(sp)$, then $(sp, read, iaddr, k, p_i) \in \mathbf{T}_{\mathcal{M}}$, which means the result $p_i$ is enforced from a valid memory access log table. Similarly, suppose that witness $w_i$ is got from memory access of $addr_j$ with access type $ltype$ as a result of $mop_k$, then $(mem, addr_j, ltype, k, w_i) \in \mathbf{T}_{\mathcal{M}}$.  Third, we enforce that all the cells in bit column are either zero or one and all the cells in u64 witness column and operand column are in $\mathbf{T}_{64}$ (less than $2^{64}$).

\subsection{Frame Circuit}
\label{chp:frame-circuit}
Frame Circuit is a table (see Figure \ref{fig:frame-circuit}) that helps us to find out the next $iaddr$ of the return instruction (see Section \ref{chp:control-flow-ins}). Each entry of $\mathbf{T}_\mathcal{F}$ is a tuple of $(prevFrame, currentFrame, iaddr)$  where $currentFrame$ is the tid of the call instruction that starts this call frame, $prevFrame$ is the $tid$ of the call instruction of previous call frame and $iaddr$ is the call instruction address of the current call frame. Suppose that $t_i$ is a return instruction at state $s_i$ with $(currentFrame, prevFrame)$ and the state $s_{i+1} = t_i \circ t_{i-1} \circ \cdots t_0(s_0)$, then we constrain that
\[
    plookup(\mathbf{T}_\mathcal{F}, (prevFrame, currentFrame, s_{i+1}.(iaddr-1))) = 0
\] to make sure the return address is correct (see Figure \ref{fig:frame-circuit}).
\begin{figure}[!h]
\centerline{
\includegraphics[scale=0.7]{figs/frame-circuit.png}
}
\caption{Frame circuit}\label{fig:frame-circuit}
\end{figure}

\subsection{Access Log Circuit}
\label{chp:access-log-circuit}
Recall that the access log circuit is a unique table corresponding to a valid memory access log sequence and satisfies Equation \ref{eq:rw-constraints}. In WASM specification, an access log is used for three different types that are memory access, stack access and global access. Each access log has a type field that is either \emph{Init}, \emph{Read} or \emph{Write} and all logs are sorted by $(address, (tid, tmid))$ where $address$ is indexed by $ (mmid, offset)$, $tid$ is the transition index of the execution log that contains the access and $tmid$ is the index of the access micro-op in that instruction (see Table \ref{tbl:rw-table}).

\subsection{IO Circuits that Support Zero-knowledge}
Zero-knowledge of inputs is not supported in WASM specification. Thus to support the private inputs which we do not want to leak, we need to add special instructions in the \zkwasm\, to distinguish between private and public inputs. We represent public inputs in a separate column and use the polynomial lookup to link input values with the result of \emph{get\_public\_input(inputCursor)} (See Figure \ref{fig:public-input}).
\begin{figure}[!ht]
\centerline{
\includegraphics[scale=0.7]{figs/public-input.png}
}
\caption{Public input circuit}\label{fig:public-input}
\end{figure}
 Similarly, we use a separate column to hold output data and use a polynomial lookup to enforce that the value we output in the execution circuit $aux$ cell lies in the output column. When dealing with private inputs from $\emph{get\_private\_input(inputCursor)}$, we put them into the related cell with no constraints as the proof system will hide the value for us.
\section{Instruction Circuits}
Based on execution trace circuit in Section \ref{chp:architecture-circuits}, we define constraints $\mathbf{C}_{op}$ for each opcode $op$. Since the constraint defined on execution trace circuit will be applied on a row basis, and the ingredients of the constraints of each $op$ will span over multiple rows, we use notation $C(curr+k)$ to denote the $k$th cell in column $c$ followed by the current row.

For example, suppose that we want to define the constraints of add instruction using the following layout

\begin{table}[!h]
\begin{center}
\begin{tabular}{ | c | c | c | c | c | c | c | c | c | c | c | }
  \hline
  start & opcode & bit cell & state & aux & $address \in T_{I}$ & $sp \in T_\mathcal{F}$& u64 cell & extra \\ 
  \hline
   true & $add$ & $overflow$ & $tid$ & $nil$ & $iaddr_0$ & sp & $w_0$ & $nil$\\ 
 \hline
   0 & $readStack$ & $nil$ & $nil$ & $nil$ & $nil$ & $sp_0$ & $w_1$ & $nil$\\ 
 \hline
   0 & $readStack$ & $nil$ & $nil$ & $nil$ & $nil$ & $sp_1$ & $w_2$ & $nil$\\ 
 \hline
   0 & $writeStack$ & $nil$ & $nil$ & $nil$ & $nil$ & $sp_2$ & $w_3$ & $nil$\\ 
 \hline
   true & $otherop$ & -- & $tid+1$ & $nil$ & $iaddr_1$ & $sp'$ & $w_0'$ & $nil$\\
 \hline
\end{tabular}
\caption{add circuit within execution trace circuit}
\label{tbl:ex-table}
\end{center}
\end{table}
\noindent where $w_1$, $w_2$ are get from the stack and $w_2$ is equal to the result of the add instruction which is pushed back to the stack.
\begin{verbatim}
def add:=
    w0 = read(stack sp);
    sp0 = sp-1;
    w1 = read(stack sp0);
    w2 = (w1 + w0) mod 2^64;
    write(stack, sp0, w2);
    FALLTHROUGH
\end{verbatim}

First, we know that by definition of add opcode, $w_0 = (w_1+w_2) \bmod 2^{64}$. Thus encode the $\bmod$ semantic into arithmetic constraint, we get that $w_0 + overflow \times 2^{64}= w_1 + w_2$. Second we enforce the stack operation are valid, that are $(stack, read, sp_0, tid, 0, w0) \in T_\mathcal{M}$, $(stack, read, sp_1, tid, 1, w1) \in T_\mathcal{M}$ and $(stack, write, sp_2, tid, 2, w2) \in T_\mathcal{M}$. Third, we need to constraint that the next instruction must follow $iaddr_0$ in address, therefore $iadd_1 = iaddr_0 + 1$. In the end we constraint the $sp$ column by $sp_0 + 1= sp$, $sp_1 + 1= sp_0$, $sp_2 = sp+1$ and $sp' = sp_2$. Put it all together, and replase variables using the notation of $columnName.(curr + k)$, we have
\[
    \mathbf{C_{add}} = \begin{cases}
        &w.(curr) + bit.(curr) \times 2^{64} - w.(curr+1) + (w.curr + 2) = 0 \\
        &Plookup(T_\mathcal{M}, (stack, read, sp.(curr), tid, 0, w0)) = 0 \\
        &Plookup(T_\mathcal{M}, (stack, read, sp.(curr+1), tid, 1, w1)) = 0 \\
        &Plookup(T_\mathcal{M}, (stack, write, sp.(curr), tid, 2, w2) = 0 \\
        &iaddr.curr + 1 - iaddr.(curr + 4) = 0\\
        &sp.curr + 1 - sp.(curr+1) = 0\\
        &sp.(curr+4) - sp.(curr) - 1 = 0
    \end{cases}
\]
Since constraints are applied on a row basis of a circuit. We need to make sure that $\mathbf{C}_{add}$ does not apply on rows that is not a start row of a instruction block or a block with other opcode. So a natural way to apply $\mathcal{C}_{add}$ only on necessary rows is to multiply $\mathbf{C}_{add}(curr)$ with $curr.start \times (curr.opcode == op)$ and the final constraint related to opcode add is $\overline{\mathbf{C}}_{add}(curr) := curr.start \times (curr.opcode == op) \times \mathbf{C}_{add}(curr) = 0$.

After we constructed all the constraints $\mathcal{C}_{op_i}$ for all opcodes $op_i$, we simply sum them up and get the final constraint $\mathcal{C}_{op}(curr) := \sum_i cur.start \times (cur.opcode == op) \times \mathcal{C}_{op_i}(curr) = 0$.

\subsection{Numeric Instructions}
\label{chp:numeric-instruction}
Numeric Instructions are the majority instructions in WASM. In general, semantics of numeric instructions contains three parts, parameters preparation, arithmetic calculation, result writeback and FALLTHROUGH as follows.
\begin{verbatim}
def arithop :=
    param1 = read(stack sp); \\ parameters preparation
    param2 = read(stack (sp-1)); \\ parameters preparation
    ...
    paramN = read(stack (sp-N+1)); \\ parameters preparation
    result = arith(param1, param2, param3, ..., paramN); \\ calculation
    write(stack, (sp-N+1), result); \\ result write back
    sp = sp-N+1;
    FALLTHROUGH;
\end{verbatim}
Based on the arithmetic defintion, we assign the cells in the execution trace circuit $T_\mathcal{E}$ in Table \ref{tbl:arith-instruction}.
\begin{table}[!h]
\begin{center}
\begin{tabular}{ | c | c | c | c | c | c | c | c | c | c | c | }
  \hline
  start & opcode & bit cell & state & aux & $address \in T_{I}$ & $sp \in T_\mathcal{F}$& u64 cell & extra \\ 
  \hline
   true & $arithop$ & $nill$ & $tid$ & $nil$ & $iaddr_0$ & sp & $param_0$ & $nil$\\ 
 \hline
   0 & $nil$ & $nil$ & $nil$ & $nil$ & $nil$ & $nil$ & $\cdots$ & $nil$\\ 
 \hline
   0 & $nil$ & $nil$ & $nil$ & $nil$ & $nil$ & $nil$ & $param_N$ & $nil$\\ 
 \hline
   0 & $nil$ & $nil$ & $nil$ & $nil$ & $nil$ & $nil$ & $result$ & $nil$\\ 
 \hline
    true & $otherop$ & -- & $tid+1$ & $nil$ & $iaddr_1$ & $sp'$ & $w_0'$ & $nil$\\
 \hline
\end{tabular}
\caption{add circuit within execution trace circuit}
\label{tbl:arith-instruction}
\end{center}
\end{table}
Therefore, assume the constraint is applied on the first row of the instruction block and rewrite the constraint of arithop using the notation of $columnName.(curr+k)$ form, we get
\[
    \mathbf{C_{arith}} = \begin{cases}
        &arith(w.(curr), w.(curr+1), ..., w.(curr+N-1)-w.(curr+N) = 0 \\
        &Plookup(T_\mathcal{M}, (stack, read, sp.(curr)-k, tid, k, w.(curr+k)) = 0 \\
        &Plookup(T_\mathcal{M}, (stack, write, sp'.(curr)-N+1, tid, N, w.(curr+N)) = 0 \\
        &iaddr.curr + 1 - iaddr.(curr + 4) = 0\\
        &sp.curr - sp' - N + 1 = 0\\
    \end{cases}
\]
\subsection{Control Flow Instructions}
In WASM specification, there three different type of control flow: FALLTHROUGH, branch, and call (return). Implementation of the FALLTHROUGH is already covered in Section \ref{chp:numeric-instruction}. Thus it is sufficient to implement call(return) and branch.

\smallskip\noindent\emph{Call (Return) Circuit.}
Call instruction will first add a new Frame Table Entry $(tid, frameId, iaddr_0)$ into the Frame Circuits $T_\mathcal{F}$ and then loading calling parameters onto the stack and goto the $iaddr_1$ for next instruction (see Table \ref{tbl:call-instruction} for the circuit layout of \emph{call}).
\begin{table}[!h]
\begin{center}
\begin{tabular}{ | c | c | c | c | c | c | c | c | c | c | c | }
  \hline
  start & opcode & bit cell & state & aux & $address \in T_{I}$ & $sp \in T_\mathcal{F}$& u64 cell & extra \\ 
  \hline
   true & $call(iaddr_1)$ & $nill$ & $tid$ & $nil$ & $iaddr_0$ & sp & $w_0$ & $nil$\\ 
 \hline
   0 & $nil$ & $nil$ & $prevFrameId$ & $nil$ & $nil$ & $nil$ & $w_1$ & $nil$\\ 
 \hline
   0 & $nil$ & $nil$ & $nil$ & $nil$ & $nil$ & $nil$ & $w_2$ & $nil$\\ 
 \hline
   0 & $nil$ & $nil$ & $nil$ & $nil$ & $nil$ & $nil$ & $w_3$ & $nil$\\ 
 \hline
   true & $otherop$ & -- & $tid + 1$ & $nil$ & $iaddr_1$ & $sp'$ & $w_0'$ & $nil$\\
 \hline
   0 & $nil$ & $nil$ & $newFrameId = tid$ & $nil$ & $nil$ & $nil$ & $w_3$ & $nil$\\ 
 \hline
\end{tabular}
\caption{circuit layout of call}
\label{tbl:call-instruction}
\end{center}
\end{table}
where the circuit constraint is:
\[
    C_{call} 
\]

 As we mentioned in Section \ref{chp:frame-circuits}, all the entries in $T_\mathcal{F}$ is used to help the return instruction to find the correct calling frame so that we can define the semantic of \emph{return} by finding the correct \emph{returnIaddr} in $T_\mathcal{F}$ (see Table \ref{tbl:return-instruction} for the circuit layout of \emph{return}).
\begin{table}[!h]
\begin{center}
\begin{tabular}{ | c | c | c | c | c | c | c | c | c | c | c | }
  \hline
  start & opcode & bit cell & state & aux & $address \in T_{I}$ & $sp \in T_\mathcal{F}$& u64 cell & extra \\ 
  \hline
   true & $call(iaddr_1)$ & $nill$ & $tid$ & $nil$ & $iaddr_0$ & sp & $w_0$ & $nil$\\ 
 \hline
   0 & $nil$ & $nil$ & $prevFrameId$ & $nil$ & $nil$ & $nil$ & $w_1$ & $nil$\\ 
 \hline
   0 & $nil$ & $nil$ & $nil$ & $nil$ & $nil$ & $nil$ & $w_2$ & $nil$\\ 
 \hline
   0 & $nil$ & $nil$ & $nil$ & $nil$ & $nil$ & $nil$ & $w_3$ & $nil$\\ 
 \hline
   true & $otherop$ & -- & $tid + 1$ & $nil$ & $iaddr_1$ & $sp'$ & $w_0'$ & $nil$\\
 \hline
   0 & $nil$ & $nil$ & $newFrameId = tid$ & $nil$ & $nil$ & $nil$ & $w_3$ & $nil$\\ 
 \hline
\end{tabular}
\caption{circuit layout of return}
\label{tbl:return-instruction}
\end{center}
\end{table}
\subsection{Local Set and Get Instructions}
see Table \ref{tbl:local-setget-ins}
\begin{table}[!h]
\begin{center}
\begin{tabular}{ | c | c | c | c | c | c | c | c | c | c | c | }
  \hline
  start & opcode & bit cell & state & aux & $address \in T_{I}$ & $sp \in T_\mathcal{F}$& u64 cell & extra \\ 
  \hline
   true & $call iaddr_1$ & $nill$ & $tid$ & $nil$ & $iaddr_0$ & sp & $w_0$ & $nil$\\ 
 \hline
   0 & $nil$ & $nil$ & $prevFrameId$ & $nil$ & $nil$ & $nil$ & $w_1$ & $nil$\\ 
 \hline
   0 & $nil$ & $nil$ & $nil$ & $nil$ & $nil$ & $nil$ & $w_2$ & $nil$\\ 
 \hline
   0 & $nil$ & $nil$ & $nil$ & $nil$ & $nil$ & $nil$ & $w_3$ & $nil$\\ 
 \hline
   true & $otherop$ & -- & $tid + 1$ & $nil$ & $iaddr_1$ & $sp'$ & $w_0'$ & $nil$\\
 \hline
   0 & $nil$ & $nil$ & $newFrameId = tid$ & $nil$ & $nil$ & $nil$ & $w_3$ & $nil$\\ 
 \hline
\end{tabular}
\caption{add circuit within execution trace circuit}
\label{tbl:local-setget-ins}
\end{center}
\end{table}
\subsection{Conversion Instructions}
see Table \ref{tbp:conversion-ins}
\begin{table}[!h]
\begin{center}
\begin{tabular}{ | c | c | c | c | c | c | c | c | c | c | c | }
  \hline
  start & opcode & bit cell & state & aux & $address \in T_{I}$ & $sp \in T_\mathcal{F}$& u64 cell & extra \\ 
  \hline
   true & $call iaddr_1$ & $nill$ & $tid$ & $nil$ & $iaddr_0$ & sp & $w_0$ & $nil$\\ 
 \hline
   0 & $nil$ & $nil$ & $prevFrameId$ & $nil$ & $nil$ & $nil$ & $w_1$ & $nil$\\ 
 \hline
   0 & $nil$ & $nil$ & $nil$ & $nil$ & $nil$ & $nil$ & $w_2$ & $nil$\\ 
 \hline
   0 & $nil$ & $nil$ & $nil$ & $nil$ & $nil$ & $nil$ & $w_3$ & $nil$\\ 
 \hline
   true & $otherop$ & -- & $tid + 1$ & $nil$ & $iaddr_1$ & $sp'$ & $w_0'$ & $nil$\\
 \hline
   0 & $nil$ & $nil$ & $newFrameId = tid$ & $nil$ & $nil$ & $nil$ & $w_3$ & $nil$\\ 
 \hline
\end{tabular}
\caption{add circuit within execution trace circuit}
\label{tbl:conversion-ins}
\end{center}
\end{table}
\section{Formal Verification of Semantic Enforcement [optional]}
\section{Customized Instruction Extension}
\label{chp:foreign}
Just like architecture specific instructions in normal hardware, \zkwasm\, supports customizing foreign instruction extension as well. More specificly, \zkwasm\, provides two ways to extend the instructions set, one way is implementing customized inline opcodes and the other way is batching external proofs of pure functions.

Given a fixed image, an entry function and an array of input arguments, the execution trace is fixed which means the number of instructions are fixed. As we described in Section \ref{chp:ex-table}, each instruction occupies $n$ (a fixed number of) rows in the execution circuit $T_\mathcal{E}$. Thus the total rows of $T_\mathcal{E}$ is fixed. When doing proof in Halo2 using KZG commitment, each column is interpolated into polynomials using FFT (fast fourier transform). Because FFT is an algorithm of $NlogN$ complexity, the total rows of $T_\mathcal{E}$ affacts the overall performance in a nonlinear way. Thus to reduce the number of columns, a good way is to batch multiple instructions into one.

When the semantic of instructions we would like to batch is simple and can fit into one instruction block, we can use the inline extension. For example, suppose that we want to sum the lowest 4 bits $x:u64$ by function $sumLowest(x)$. If we use a standard loop to implement the algorithm, it will takes $4$ instructions to extract $4$ bits and takes another $2$ to do the sum. However, if we inline this function in to a foreign function, then we can encode the arithmetic constraint within one instruction block. As a case study, we compare the SHA256 execution trace with and without inline the following macro:

\begin{table}[!h]
\small
\begin{center}
\begin{tabular}{ | c | p{1cm} | c | p{1.5cm} | }
  \hline
  original & original rows & customized & optimized rows\\ 
  \hline
  $\lambda x, y, (x \& y) | (complete(x) \& z)$ & 4 & $ch(x,y)$ & 1 \\
  \hline
  $\lambda x, y, z, z | (x \& (y | z))$ & 2 & $maj(x, y, z)$ & 1\\
  \hline
  $\lambda x, rotr_{32}(x, 2) | rotr_{32}(x, 13) | rotr_{32}(x, 22)$ & 5 & $lsigma0(x)$ & 1 \\
  \hline
  $\lambda x, rotr_{32}(x, 6) | rotr_{32}(x, 11) | rotr_{32}(x, 25)$ & 5 & $lsigma1(x)$ & 1\\
 \hline
   sha256 using pure WASM instructions & 1 & sha256 using customized instructions & XXX\\
 \hline
\end{tabular}
\caption{row reduce by using customized instructions}
\label{tbl:memory-instruction}
\end{center}
\end{table}


\section{performance benchmark}
\label{chp:performance}

\bibliographystyle{plain}
\bibliography{main}

\end{document}
\endinput
%%
%% End of file `sample-acmsmall.tex'.
