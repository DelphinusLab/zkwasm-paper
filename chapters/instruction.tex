\section{Instruction Circuits}
\label{chp:instruction-circuits}
Once the architecture circuits are all prepared in Section \ref{chp:architecture-circuits}, the remaining things are constructing the circuits $\mathbf{C}_{op}$ of various opcode $op$ for instructions supported by WASM specification. Since the constraint defined on the execution trace circuit will be applied on a row basis, and the cells of the constraints of each $op$ will span over multiple rows, we use the notation $c.(curr+k)$ to denote the $k$th cell in column $c$ followed by the current row.

For example, suppose that we want to define the constraints of add instruction (see Figure \ref{fig:example-ins}) using the circuit layout in Table \ref{tbl:add-instruction}
\begin{table}[!h]
\begin{center}
\caption{Add circuit within execution trace circuit}
\label{tbl:add-instruction}
\begin{tabular}{ | c | c | c | c | c | c | c | c | c | c | c | }
  \hline
  start & opcode & bit cell & state & aux & $address \in T_{I}$ & $sp \in T_\mathcal{F}$& u64 cell & extra \\ 
  \hline
   true & $add$ & $overflow$ & $tid$ & $nil$ & $iaddr_0$ & sp & $w_0$ & $nil$\\ 
 \hline
   0 & $readStack$ & $nil$ & $nil$ & $nil$ & $nil$ & -- & $w_1$ & $nil$\\ 
 \hline
   0 & $readStack$ & $nil$ & $nil$ & $nil$ & $nil$ & -- & $w_2$ & $nil$\\ 
 \hline
   0 & $writeStack$ & $nil$ & $nil$ & $nil$ & $nil$ & -- & $w_3$ & $nil$\\ 
 \hline
   true & $otherop$ & -- & $tid+1$ & $nil$ & $iaddr_1$ & $sp'$ & $w_0'$ & $nil$\\
 \hline
\end{tabular}

\end{center}
\end{table}
where $w_1$, $w_2$ are got from the stack and $w_0$ is equal to the result of the add instruction which is pushed back to the stack.
\begin{figure}
\small
\centering
\begin{verbatim}
def add :=
    w1 = read(stack sp);
    w2 = read(stack sp-1);
    sp' = sp-1;
    w0 = (w1 + w0) mod 2^64;
    write(stack, sp-1, w0);
    FALLTHROUGH
\end{verbatim}
\caption{Add instruction definition}
\label{fig:example-ins}
\end{figure}
First, we know that by definition of $add$, $w_0 = (w_1+w_2) \bmod 2^{64}$. Thus by introducing a new witness $overflow$ we encode the $\bmod$ semantic into arithmetic constraint as $w_0 + overflow \times 2^{64}= w_1 + w_2$. Second, we enforce the stack operation are valid, that is $(stack, read, sp-1, tid, 0, w0) \in T_\mathcal{M}$, $(stack, read, sp, tid, 1, w1) \in T_\mathcal{M}$ and $(stack, write, sp-1, tid, 2, w2) \in T_\mathcal{M}$. Third, we need to constrain that the next instruction must follow $iaddr_0$ in address, therefore $iadd_1 = iaddr_0 + 1$. In the end, we constrain the $sp$ column by $sp' + 1= sp$. Put it all together, and replace variables using the notation of $columnName.(curr + k)$, we have Equation \ref{eq:cs-add}.
\begin{equation}
    \mathbf{C_{add}} = \begin{cases}
        &w.(curr) + bit.(curr) \times 2^{64} - w.(curr+1) + (w.curr + 2) = 0 \\
        &Plookup(T_\mathcal{M}, (stack, read, sp.(curr), tid, 0, w1)) = 0 \\
        &Plookup(T_\mathcal{M}, (stack, read, sp.(curr)-1, tid, 1, w2)) = 0 \\
        &Plookup(T_\mathcal{M}, (stack, write, sp.(curr)-1, tid, 2, w0) = 0 \\
        &iaddr.curr + 1 - iaddr.(curr + 4) = 0\\
        &sp.(curr+4) + 1 - sp.(curr)  = 0
    \end{cases}
    \label{eq:cs-add}
\end{equation}
Since constraints are applied on a row basis of a circuit, we need to make sure that $\mathbf{C}_{add}$ does not apply on rows that are not a starting row of an instruction block or a block with other opcodes. So a natural way to apply $\mathcal{C}_{add}$ only on necessary rows is to multiply $\mathbf{C}_{add}(curr)$ with $curr.start \times (curr.opcode == op)$ and the final constraint related to opcode add is $\overline{\mathbf{C}}_{add}(curr) := curr.start \times (curr.opcode == op) \times \mathbf{C}_{add}(curr) = 0$. 

\begin{remark}
For better readability, from this point we will simply use the name of the cell instead of the notation $columnName.(curr + k)$ if no confusion is introduced by doing so.
\end{remark}

\noindent The content of the rest of this section is arranged in subsections to describe circuits of instructions in different categories. After we have constructed all the constraints $\mathcal{C}_{op_i}$ for all opcodes $op_i$, we simply sum them up and get the final constraint $\mathcal{C}_{op}(curr) := \sum_i cur.start \times (cur.opcode == op) \times \mathcal{C}_{op_i}(curr) = 0$ for $\mathbf{T}_\mathcal{E}$.


\subsection{Numeric Instructions}
\label{chp:numeric-instruction}
Numeric Instructions are the majority of instructions in WASM. In general, the semantics of numeric instructions contain file parts, parameters preparation, arithmetic calculation, writing result back to stack, update stack pointer and FALLTHROUGH as in Figure \ref{fig:ins-arith}.
\begin{figure}[h!]
\small
\centering
\begin{verbatim}
def arithop :=
    param1 = read(stack sp); \\ parameters preparation
    param2 = read(stack (sp-1)); \\ parameters preparation
    ...
    paramN = read(stack (sp-N+1)); \\ parameters preparation
    result = arith(param1, param2, param3, ..., paramN); \\ calculation
    write(stack, (sp-N+1), result); \\ result write back
    sp = sp-N+1;
    FALLTHROUGH;
\end{verbatim}
\caption{Arithmetic instruction}
\label{fig:ins-arith}
\end{figure}

Based on the arithmetic definition, we assign the cells in the execution trace circuit $\mathbf{T}_\mathcal{E}$ in Table \ref{tbl:arith-instruction} and the constraints of arithmetic opcode are defined in Equation \ref{eq:cs-arith}.
\begin{table}[!h]
\small
\begin{center}
\caption{Add circuit in execution trace circuit}
\label{tbl:arith-instruction}
\begin{tabular}{ | c | c | c | c | c | c | c | c | c | c | c | }
  \hline
  start & opcode & bit cell & state & aux & $address \in T_{I}$ & $sp \in T_\mathcal{F}$& u64 cell & extra \\ 
  \hline
   true & $arithOp$ & $nill$ & $tid$ & $nil$ & $iaddr_0$ & sp & $param_0$ & $nil$\\ 
 \hline
   0 & $nil$ & $nil$ & $nil$ & $nil$ & $nil$ & $nil$ & $\cdots$ & $nil$\\ 
 \hline
   0 & $nil$ & $nil$ & $nil$ & $nil$ & $nil$ & $nil$ & $param_N$ & $nil$\\ 
 \hline
   0 & $nil$ & $nil$ & $nil$ & $nil$ & $nil$ & $nil$ & $result$ & $nil$\\ 
 \hline
    true & $otherop$ & -- & $tid+1$ & $nil$ & $iaddr_1$ & $sp'$ & $w_0'$ & $nil$\\
 \hline
\end{tabular}

\end{center}
\end{table}
\begin{equation}
    \mathbf{C}_{arith} = \begin{cases}
        &arith(param_0, param_1, ..., param_N) - result = 0 \\
        &Plookup(\mathbf{T}_\mathcal{M}, (stack, read, sp - k, tid, k, param_k) = 0 \\
        &Plookup(\mathbf{T}_\mathcal{M}, (stack, write, sp' - 1, tid, N, result) = 0 \\
        &iaddr_0 + 1 - iaddr_1 = 0\\
        &sp - sp' - N + 1 = 0\\
    \end{cases}
\label{eq:cs-arith}
\end{equation}
\subsection{Control Flow Instructions}
\label{chp:control-flow-ins}
In WASM specification, there are three different categories of control flow: \emph{FALLTHROUGH}, \emph{branch}, and \emph{call} (\emph{return}). Implementation of the \emph{FALLTHROUGH} is already covered in Section \ref{chp:numeric-instruction}. Thus it is sufficient to implement call (return) and branch.\\

\noindent\emph{Call (Return) Circuit.}
Call instruction will first add a new frame table entry $(tid, preFrameId, iaddr_0)$ into the frame circuits $\mathbf{T}_\mathcal{F}$ and then load calling parameters onto the stack and go to the $iaddr_1$ for next instruction (see Table \ref{tbl:call-instruction} for the circuit layout of \emph{call}). The circuit constraint for call instruction is Equation \ref{eq:cs-call}.
\begin{table}[!h]
\small
\begin{center}
\caption{Circuit layout of call}
\label{tbl:call-instruction}
\begin{tabular}{ | c | c | c | c | c | c | c | c | c | c | c | }
  \hline
  start & opcode & bit cell & state & aux & $address \in T_{I}$ & $sp \in T_\mathcal{F}$& u64 cell & extra \\ 
  \hline
   true & $call(targetIaddr)$ & $nill$ & $tid$ & $nil$ & $iaddr_0$ & sp & $param_0$ & $nil$\\ 
 \hline
   0 & $nil$ & $nil$ & $preFrameId$ & $nil$ & $nil$ & $nil$ & $\cdots$ & $nil$\\ 
 \hline
   0 & $nil$ & $nil$ & $nil$ & $nil$ & $nil$ & $nil$ & $param_N$ & $nil$\\ 
 \hline
   true & $otherop$ & -- & $tid + 1$ & $nil$ & $iaddr_1$ & $sp'$ & $nil$ & $nil$\\
 \hline
   0 & $nil$ & $nil$ & $tid$ & $nil$ & $nil$ & $nil$ & $nil$ & $nil$\\ 
 \hline
\end{tabular}

\end{center}
\end{table}
\begin{equation}
    C_{call} = \begin{cases}
        &Plookup(\mathbf{T}_\mathcal{M}, (stack, write, sp+i, tid, i, param_i)) = 0 \\
        &Plookup(\mathbf{T}_\mathcal{F}, (tid, pFrameId, iaddr_0)) = 0 \\
        &iaddr_1 - targetIaddr = 0\\
        &sp' - sp - N = 0 \\
        &nFrameId - tid = 0.
    \end{cases}
\label{eq:cs-call}
\end{equation}
 To define the constraint for \emph{Return} instruction, we need to find the correct return address of the current frame and set the frame state to the previous frame. Recall that, as described in Section \ref{chp:frame-circuit}, the entries in $\mathbf{T}_\mathcal{F}$ are used to help the return instruction to find the correct calling frame and previous frame. Thus we can define the semantics of \emph{return} by finding the correct return \emph{iaddr} from $T_\mathcal{F}$ and then enforce that the next instruction address is equal to \emph{iaddr} and update the frame state accordingly (see Table \ref{tbl:return-instruction} for the circuit layout and Equation \ref{eq: cs-return} for the circuit constraint).
\begin{table}[!h]
\begin{center}
\caption{Circuit layout of return}
\label{tbl:return-instruction}
\begin{tabular}{ | c | c | c | c | c | c | c | c | c | c | c | }
  \hline
  start & opcode & bit cell & state & aux & $address \in \mathbf{T}_{I}$ & $sp \in \mathbf{T}_\mathcal{F}$& u64 cell & extra \\ 
  \hline
   true & $return$ & $nill$ & $tid$ & $nil$ & $iaddr_0$ & sp & $nil$ & $nil$\\ 
 \hline
   0 & $nil$ & $nil$ & $prevFrameId$ & $nil$ & $nil$ & $nil$ & $nil$ & $nil$\\ 
 \hline
   true & $otherop$ & -- & $tid + 1$ & $nil$ & $iaddr_1$ & $sp'$ & $nil$ & $nil$\\
 \hline
   0 & $nil$ & $nil$ & $nFrameId$ & $nil$ & $nil$ & $nil$ & $nil$ & $nil$\\ 
 \hline
\end{tabular}

\end{center}
\end{table}
\begin{equation}
    C_{return} =  \begin{cases}
        &Plookup(\mathbf{T}_\mathcal{F}, (pFrameId, nFrameId, iaddr_1 - 1)) = 0 \\
        &sp' - sp = 0 \\
    \end{cases}
\label{eq: cs-return}
\end{equation}
\\
\noindent\emph{Branch Circuit.}
Branch instructions in WASM include \emph{br}, \emph{br\_if}, \emph{if * then * else *}, etc. The semantics of branch instructions can be uniformly abstracted as three steps (see Figure \ref{fig:inst-branch}).
\begin{figure}[h!]
\small
\centering
\begin{verbatim}
def branchop :=
    param1 = read(stack sp); \\ parameters preparation
    param2 = read(stack (sp-1)); \\ parameters preparation
    ...
    paramN = read(stack (sp-N+1)); \\ parameters preparation
    iaddr1 = select(param1, param2, ..., paramN); \\ calculate branch address
    GOTO iaddr2; \\branch to target address
\end{verbatim}
\caption{Semantic of branch instruction}
\label{fig:inst-branch}
\end{figure}
\smallskip The circuit layout of the branch instruction is sketched in Table \ref{tbl:branch-instruction} and its circuit constraint is defined in Equation \ref{eq:cs-branchop}. 
\begin{table}[!h]
\small
\begin{center}
\caption{Circuit layout of call}
\label{tbl:branch-instruction}
\begin{tabular}{ | c | c | c | c | c | c | c | c | c | c | c | }
  \hline
  start & opcode & bit cell & state & aux & $address \in T_{I}$ & $sp \in T_\mathcal{F}$& u64 cell & extra \\ 
  \hline
   true & $branchop$ & $nill$ & $tid$ & $nil$ & $iaddr_0$ & sp & $param_0$ & $nil$\\ 
 \hline
   0 & $nil$ & $nil$ & $frameId$ & $nil$ & $nil$ & $nil$ & $\cdots$ & $nil$\\ 
 \hline
   0 & $nil$ & $nil$ & $nil$ & $nil$ & $nil$ & $nil$ & $param_N$ & $nil$\\ 
 \hline
   true & $otherop$ & -- & $tid + 1$ & $nil$ & $iaddr_1$ & $sp'$ & $nil$ & $nil$\\
 \hline
   0 & $nil$ & $nil$ & $frameId$ & $nil$ & $nil$ & $nil$ & $nil$ & $nil$\\ 
 \hline
\end{tabular}

\end{center}
\end{table}
\begin{equation}
    C_{branch} = \begin{cases}
        &Plookup(T_\mathcal{M}, (stack, write, sp+i, tid, i, param_i)) = 0 \\
        &iaddr_1 - select(param_0, param_1, \cdots) = 0 \\
        &nFrameId - pFrameId = 0.
    \end{cases}
\label{eq:cs-branchop}
\end{equation}

\subsection{Memory (Stack, Global) Instructions}
Memory, Stack and Global instructions can be abstracted as a tuple of ($category$, $type$, $address$, $size$ = 8|16|32|64, $value$) where $category$ can be Memory, Stack or Global and $type$ can be Init, Read or Write. The layout of the circuit is defined in Table \ref{tbl:memory-instruction}. By using the access log circuit defined in Chapter \ref{chp:access-log-circuit}, the constraint for the memory circuit is simply $(category, ltype, tid, address, value') \in T_\mathcal{M} \wedge trunc(value', size) = value$.

\begin{remark}
For read, this constraint ensures the result read from $address$ is valid. For write, $T_\mathcal{M}$ ensures that the next $read$ of $address$ will return the previously written value correctly.
\end{remark}
\begin{table}[!h]
\small
\begin{center}
\caption{Memory access circuit within execution trace circuit}
\label{tbl:memory-instruction}
\begin{tabular}{ | c | c | c | c | c | c | c | c | c | c | c | }
  \hline
  start & opcode & bit cell & state & aux & $address \in T_{I}$ & $sp \in T_\mathcal{F}$& u64 cell & extra \\ 
  \hline
   true & $op(type, size)$ & $nill$ & $tid$ & $nil$ & $iaddr_0$ & sp & $address$ & $nil$\\ 
 \hline
   0 & $nil$ & $nil$ & $frameId$ & $nil$ & $nil$ & $nil$ & $value$ & $nil$\\ 
 \hline
   true & $otherop$ & -- & $tid + 1$ & $nil$ & $iaddr_1$ & $sp'$ & $w_0'$ & $nil$\\
 \hline
   0 & $nil$ & $nil$ & $frameId = tid$ & $nil$ & $nil$ & $nil$ & $w_3$ & $nil$\\ 
 \hline
\end{tabular}

\end{center}
\end{table}